
% vim: set tw=79 sw=3 expandtab:

\chapter{Distance}
\label{ch:distance}

A point, as introduced in high-school geometry, is a particular, pure location,
without any \emph{extent}.  An object such as
\begin{itemize}[noitemsep]
   \item a curve (a one-dimensional space),
   \item a surface (a two-dimensional space),
   \item a proper space (a three-dimensional space), or
   \item a hyperspace (a space with four or more dimensions)
\end{itemize}
has points in it.

An extent is an unbroken \emph{length} corresponding to a thing's size.  What I
call ``length'' is the principle of extension.  A geometric object might be
extended in multiple ways, as in width and in height.  Each of the width and
the height is fundamentally a length, though each is a length in a different
dimension.  For example, consider a 180-degree, circular arc of one-meter
radius in the Euclidean plane.  There is a length along the arc.  The arc also
has some extension in each of the two dimensions of the plane.
\begin{figure}
   \begin{center}
   \begin{asy}
      defaultpen(fontsize(8pt));
      settings.prc = false;
      size(8cm,0);
      import three;
      real step = pi/12;
      triple p00 = cos( 0*step)*X + sin( 0*step)*Y;
      triple p01 = cos( 1*step)*X + sin( 1*step)*Y;
      triple p02 = cos( 2*step)*X + sin( 2*step)*Y;
      triple p03 = cos( 3*step)*X + sin( 3*step)*Y;
      triple p04 = cos( 4*step)*X + sin( 4*step)*Y;
      triple p05 = cos( 5*step)*X + sin( 5*step)*Y;
      triple p06 = cos( 6*step)*X + sin( 6*step)*Y;
      triple p07 = cos( 7*step)*X + sin( 7*step)*Y;
      triple p08 = cos( 8*step)*X + sin( 8*step)*Y;
      triple p09 = cos( 9*step)*X + sin( 9*step)*Y;
      triple p10 = cos(10*step)*X + sin(10*step)*Y;
      triple p11 = cos(11*step)*X + sin(11*step)*Y;
      triple p12 = cos(12*step)*X + sin(12*step)*Y;
      path3 a = arc(c=O, v1=p00, v2=p12);
      currentprojection=orthographic(camera=Z);
      draw(a);
      draw(p00--p06--p12, blue+linetype(new real[]{8,8,1,8}));
      draw(p00--p04--p08--p12, magenta+linetype(new real[]{8,8}));
      draw(p00--p03--p06--p09--p12, darkgreen+linetype(new real[]{4,4,1,4}));
      draw(p00--p02--p04--p06--p08--p10--p12, red+linetype(new real[]{4,4}));
   \end{asy}
   \end{center}
   \caption{An arc and approximations to its length.}
   \label{fig:arc}
\end{figure}
See Figure~\ref{fig:arc}.

\section{Shortest Path}

Between any two points in a space, there is a special length called a
\emph{distance}; contained in this idea is, as it turns out, most of what we
mean by ``space.'' The distance between any two points in a space is the length
of the shortest path between them.

For a one-dimensional space, the existence of a distance between every pair of
points implies that the space itself is a single, unbroken extent.

For a space of two or more dimensions, however, a distance between every pair
of points implies that there are infinitely many unbroken paths between the two
points and that one of those paths has the shortest length.

\section{Length Along Curve}

The length along a curve is the limit of a sequence of sums, each an
approximation better than the previous.  See Figure~\ref{fig:arc}.  The length
along the arc in the Euclidean plane is the limit of a sequence of approximate
sums.  The sum of the red distances is an approximation better than the sum of
the blue distances.  The \emph{smooth} arc\footnote{%
   A curve is smooth if, for any two points sufficiently close together along
   the curve, the difference between
   \begin{itemize}[noitemsep]
      \item their separation distance and
      \item the length between them along the curve
   \end{itemize}
   is so small as desired.%
}
is not composed of any straight paths.  Rather, the \emph{length} of the curved
arc is the limit of a sequence of sums of straight-path distances.  No finite
sum of straight-path distances between points along the arc is equal to the
length of the arc, but the limit of the sequence of sums can be found.  Each
further sum in the sequence is the sum of a larger number of smaller distances.
So long as the curve be smooth, the limit of the sequence of the sums is the
length of the curve.\footnote{%
   The limit for the length of a curved path exists so long as the curve be
   \emph{piece-wise smooth}.  That is, so long as the curve be continuous and
   consist of contiguous sections, each of finite length.%
}

\section{Principle of Extension}

Any length is either a sum of distances or the limit of a sequence of such
sums. A surface area, too, ultimately depends on distance because it is a
product of lengths.  So, too, for a volume of space.  The idea of area and the
idea of volume are built upon the idea of length and therefore, ultimately,
upon the idea of distance.

Regardless of the number of dimensions a space might have, it can support the
basic notion of extended geometry only if there be a unique \emph{distance}
between every pair of points in the space.  A path of definite length is not
merely a collection of points but also a rule for associating a unique distance
between every pair of points on the path.  The idea of distance is the
principle of continuous, unbroken geometry.  A continuous length is properly
divided not into points but rather into infinitesimal, contiguous elements of
distance.\footnote{%
   There is a problem that can arise from imagining a curve (for example) as
   \emph{merely} a collection of points.  Because a point has no length, no
   point can touch another point.  Regardless of how close together two points
   might be, each has zero length, and so there is a gap between them.  Because
   no point can touch another point, there is no arrangement of points, lying
   in unbroken contact, such that they constitute a length.  The idea of an
   unbroken length, a continuously extended path, is more fundamental than the
   idea of a point:  A point is a pure location \emph{in} or \emph{on}
   something, such as a curve, a surface, or proper space.  That thing must
   exist first, in order for there then to be points identified on it, and it
   must define what the distance between a pair of points is.  Beginning in the
   20th Century, a geometric object has often been considered as essentially a
   collection of points.  This can be reasonable only if one remember that
   there is also, somehow associated with the definition of the points, the
   idea of a \emph{distance} between every pair of points.%
}
This idea is what enables integral calculus to work.

\section{Geodesic}

