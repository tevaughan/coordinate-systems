
% vim: set tw=79 sw=3 expandtab:

\chapter{Prolegomenon}

In the present book I work through some of the basics of differential geometry.
Also, for coordinate systems and component transformations in Euclidean space,
I develop an approach that explicitly treats a certain kind of vector---for
example, a displacement vector---as simple and independent of any coordinate
system.  This is inspired by ideas that Neil Huffaker introduced in his
graduate lectures on electrodynamics at the University of Oklahoma in the early
1990s.  The material should be accessible to anyone who has studied
calculus-based physics.

The title is ``Coordinate Systems in Physical Space,'' and a brief reflection
on the choice of title provides an introduction to the material.

\section{Coordinate Systems}

First of all, the book is about coordinate systems.  But what does
``coordinate'' mean?

\subsection{Coordinate}

In mathematics, a coordinate is typically a real number that constrains the
location of a point in a \emph{manifold}.  A manifold is an $n$-dimensional
space,\footnote{%
   We shall typically consider cases in which $n \in \{1, 2, 3, 4\}$.%
}
every sufficiently small region of which resembles Euclidean space so well as
desired.  From now on, we shall not use the word ``manifold'' but instead just
use the word ``space'' to mean the same thing: a \emph{locally Euclidean,
mathematical space}.  (We shall below, however, distinguish a mathematical
space from a physical space.)

\subsection{Coordinate Tuple and Coordinate System}

A point in an $n$-dimensional space can be uniquely identified by a
\emph{tuple}, or ordered set, of $n$ coordinates; that is, a coordinate
$n$-tuple.  A \emph{coordinate system} is a continuous, one-to-one map from the
set of coordinate $n$-tuples onto the points in the space.\footnote{%
   Continuity implies that an infinitesimal change in any coordinate must never
   change the mapped point by a finite distance.  The one-to-one nature means
   that for every point in the space, there is exactly one tuple corresponding
   to it.%
}
Suppose that the first coordinate in the tuple $(x_1, \ldots, x_n)$ is $x_1$,
and the last is $x_n$.  Then we can specify any coordinate as $x_i$, where $i
\in \{1, 2, 3, \ldots, n\}$.

\subsection{Coordinate Line}

Allowing the $i$th coordinate to vary, while holding every other coordinate at
fixed value, produces tuples for points along a \emph{coordinate line} for the
$i$th coordinate.

For example, the surface of a sphere is a two-dimensional space.  Each of the
latitude and the longitude is a coordinate.  These coordinates form a
two-tuple, or ordered pair, $(\alpha,\beta)$, where $\alpha$ is the latitude
and $\beta$ the longitude.\footnote{%
   In order to preserve the one-to-one nature of the map from coordinates to
   points, one restricts the domain.  For example, one might choose $-90^\circ
   \leq \alpha \leq +90^\circ$ and $-180^\circ \leq \beta < +180^\circ$ as
   definitive of the domain.%
}
\begin{figure}
   \begin{center}
   \begin{asy}
      defaultpen(fontsize(8pt));
      settings.prc = false;
      size(8cm,0);
      import three;
      draw(unitsphere, white);
      path3 equator = circle(c=O, r=1.001, normal=Z);
      path3 par     = circle(c=O+1/sqrt(2)*Z, r=1/sqrt(2)+0.001, normal=Z);
      path3 prime   = circle(c=O, r=1.001, normal=X);
      path3 mer     = circle(c=O, r=1.001, normal=cos(30)*X+sin(30)*Y);
      draw(equator, red);
      draw(par, red);
      draw(prime, yellow);
      draw(mer, yellow);
   \end{asy}
   \end{center}
   \caption{Coordinate lines on the surface of a sphere.}
   \label{fig:sphere}
\end{figure}
See Figure~\ref{fig:sphere}.
\begin{itemize}
   \item By holding the latitude fixed, one finds that the set of longitudes
      maps to a \emph{parallel}, a coordinate line for the longitude, on the
      surface of the sphere.  Every path of constant latitude on the sphere is
      called a ``parallel'' because no parallel intersects any other parallel.
      Among the parallels, only the equator is a great circle; that is, the
      parallel at zero latitude is the only parallel circle centered on the
      center of the sphere.  In the figure, the equator and a northern parallel
      appear in red.
   \item Every path of constant longitude---that is, every coordinate line for
      the latitude---is a \emph{meridian} and intersects every other meridian,
      both at the north pole and at the south pole.  Every meridian is a great
      circle.  In the figure, two meridians appear in yellow.
\end{itemize}

\subsection{Value of Coordinate}

In my treatment, a coordinate is not necessarily a real number.  It might
instead be an irreducibly physical quantity, like a length.  A physical
quantity is not a number, but it can be expressed as the product of a number
and a standard unit, such as an inch.  What is generally required is that a
coordinate be an element of a simply ordered set.\footnote{%
   A set $S$ is simply ordered if, for every $a, b, c \in S$,
   \begin{itemize}[noitemsep]
      \item $a \leq b$ and $b \leq a$ imply that $a = b$;
      \item $a \leq b$ and $b \leq c$ imply that $a \leq c$; and
      \item either $a \leq b$ or $b \leq a$.
   \end{itemize}%
}
The real numbers are simply ordered, and so is the set of all lengths of the
form, $x = r \; \text{m}$, where $r$ is a real number, and ``$\text{m}$''
represents a meter.

\subsection{Arbitrary Choice of Coordinate System}

No particular coordinate system need be chosen for a given space, and so the
choice of coordinate system is arbitrary.  However, one coordinate system might
fit a space more naturally than another would fit it.  For example, the most
natural coordinate systems on the surface of an oblate
spheroid\footnote{%
   There is no preferred orientation for a grid of latitude and longitude on a
   perfect sphere.  An oblate spheroid is obtained by a scaling transformation
   that reduces the size of the sphere along a single direction.  The shape of
   the Earth is approximated by an oblate spheroid better than by a perfect
   sphere.%
}
would have the equator on the longest circumference of the spheroid.

\subsection{Cartesian Coordinate System}

There is a special kind of coordinate system, a \emph{Cartesian} coordinate
system.  In a Cartesian coordinate system, the shortest path between every pair
of points on every coordinate line lies on that coordinate line.  Not every
space admits a Cartesian coordinate system.  For example, the surface of a
sphere does not admit a Cartesian coordinate system.\footnote{%
   On the surface of the sphere, the shortest path between two distinct points
   along a parallel usually does not lie along the parallel.  The shortest path
   always lies along a great circle, and, except for the equator, a parallel is
   not a great circle.%
}
In fact, only a \emph{Euclidean} space admits a Cartesian coordinate system.
We shall see that what makes a Euclidean space special is that it has zero
\emph{curvature}.

\subsection{Transformation of Coordinate System}

We shall spend some time discussing the difference between a quantity that
depends on the choice of coordinate system and a quantity that does not depend
on the choice.  For a quantity that does depend on the coordinate system, we
shall explore, particularly for Cartesian coordinates, the different ways in
which a quantity can transform as the coordinate system changes.

\section{Physical Space}

A mathematical model of physical space is not physical space itself.  A model,
such as Euclidean space or the space-time of general relativity, is an abstract
thing in the mind.  Physical space is the space of direct, sensory experience.
Euclidean space is often a good model of physical space, but not always.  Each
of the models that I develop makes explicit what might be only implicit in a
standard mathematical treatment of space.  I do this so that each model, while
still essentially abstract, is concrete enough to be falsifiable like a
scientific theory and so that the model accords well with physical intuition.

\subsection{Explicit Principles}

Each of the models that I develop is based on two key principles.

First, a distance is not reduced to a number; rather, distance remains an
intuitive, undefined, physical, non-numeric quantity, expressed in terms of a
fundamental unit, such as the inch or the meter.  Nevertheless, a distance can
be multiplied by a number.

Second, neither a point in space nor a displacement vector between two points
is reduced to a tuple of quantities.  Rather, each of the point and the
displacement vector is regarded as both simple\footnote{%
   What is simple is not composed of parts; that is, not complex.  A tuple of
   coordinates is complex in that it has parts (coordinates).  A coordinate
   system maps a tuple to a point in space, but the point has no parts.  The
   point is simply a location.%
}
and logically prior to any coordinate system.

\subsection{Euclidean and Non-Euclidean Models}

As a global physical hypothesis, Euclidean geometry has been ruled out by
experiment.\footnote{%
   Experiments in lunar ranging from the surface of the Earth and data from the
   spacecraft, Gravity Probe B, have directly and repeatably measured the
   curvature of space near the Earth. The results are inconsistent with
   Euclidean physical space.  \citep{gpb-2011, miller-geodetic}
}
However, even in the non-Euclidean geometries of modern physics, the Euclidean
ideal is a better and better approximation as the size of the region of space
under consideration becomes smaller and smaller.  While Euclidean geometry as a
physical hypothesis applied globally has been ruled out, Euclidean geometry as
applied separately to every sufficiently small region of physical space has not
been ruled out.  In fact, the space of general relativity, which has not (yet)
been ruled out, is called ``a locally Euclidean space.''

Anyway, most of the narrative of the book is set in the practically Euclidean,
three-dimensional space of ordinary experience, but the non-Euclidean,
two-dimensional space of the surface of the sphere is also explored.  Also, we
shall at least briefly discuss both special relativity and general relativity.

