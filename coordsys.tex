
\documentclass{tufte-book}


% vim: set filetype=tex:

% Prints an epigraph and speaker in sans serif, all-caps type.
\newcommand{\openepigraph}[2]{%
  {
  \sffamily
  \noindent\textit{#1} % epigraph
  \ \\%
  \ \\%
  \noindent{#2} % author
  }
}


\usepackage{amsmath} % Allow use of $\text{}$.
\usepackage{amsthm} % theorem
\usepackage{amsfonts}
\usepackage[rgb]{xcolor} % might need loading before asymptote
\usepackage{asymptote} % figures
\usepackage[greek,english]{babel} % language selection
\usepackage{booktabs} % \toprule, \midrule, \bottomrule
\usepackage[margin=10pt,font={sl},labelfont=bf]{caption} % figure captions
\usepackage{enumitem} % for compact list with [noitemsep]
\usepackage{fancyhdr}
\usepackage{framed} % environment for framed box
\usepackage{graphicx}
\usepackage{makeidx}
\usepackage{natbib}
\usepackage{times}
\usepackage{vmargin}

% Must be last.
\usepackage[colorlinks=true,citecolor=blue,hyperfootnotes=false]{hyperref}

\makeindex % Generate the index.

% uncomment if you prefer colored hyperlinks (e.g., for onscreen viewing)
\hypersetup{colorlinks}

% Babel setup. I want the polutoniko version of Greek for the rich set of
% accents needed properly to quote from John's gospel.
\languageattribute{greek}{polutoniko}

% Amsthm setup.
% Empty argument to newtheoremstyle leaves default.
\newtheoremstyle{mytheorem}%                   name
{\topsep}%                                     space above
{\topsep}%                                     space below
{\slshape}%                                    body font
{0pt}%                                         indent
{\bfseries}%                                   head font
{\ }%                                          head punctuation
{5pt plus 1pt minus 1pt}%                      head space
{\thmname{#1}\thmnumber{\ #2}:\thmnote{\ #3}}% head spec
\theoremstyle{mytheorem}
\newtheorem{exercise}{Exercise}[section]
\newtheorem{definition}{Definition}[chapter]
\newtheorem{postulate}{Postulate}[chapter]
\newtheorem{lemma}{Lemma}[chapter]
\newtheorem{theorem}{Theorem}[chapter]

\selectlanguage{english} % Configure babel.

% vmargin setup
\setpapersize{USletter}
\setmarginsrb%
{0.375in}%           left
{0.375in}%           top
{0.375in}%           right
{0.5in}%             bottom
{3\baselineskip}%    headheight
{2\baselineskip}%    headsep
{3\baselineskip}%    footheight
{4\baselineskip}%    footskip

% mydate macro
\newcommand{\mydate}{%
   \number\year\space%
   \ifcase\month\or%
      Jan\or\ Feb\or\ Mar\or\ Apr\or\ May\or\ Jun\or%
      Jul\or\ Aug\or\ Sep\or\ Oct\or\ Nov\or\ Dec
   \fi\space%
   \number\day%
}

\newcommand{\doctitle}{Coordinate Systems in Physical Space}
\newcommand{\docsubtitle}{}
\newcommand{\docauthor}{Thomas E.~Vaughan}
\newcommand{\docsite}{\url{http://github.com/tevaughan/coordinate-systems}}



\begin{document}


% vim: set filetype=tex expandtab shiftwidth=3:

\frontmatter

% r.1 blank page
\blankpage

% v.2 epigraphs
\newpage\thispagestyle{empty}
\openepigraph{%
Wheeeeeeeeeeeeeeeeeeee.
}{%
Neil Huffaker, when writing an interesting mathematical result onto the board
during a lecture
}
\vfill
\openepigraph{%
Well, that's unfortunate.
}{%
Jack Cohn, when---after a long, difficult, and messy attempt at a derivation on
the board---arriving at the wrong result during a lecture
}
\vfill

% r.3 full title page
\maketitle

% v.4 copyright page
\newpage
\begin{fullwidth}
~\vfill
\thispagestyle{empty}
\setlength{\parindent}{0pt}
\setlength{\parskip}{\baselineskip}
Copyright \copyright\ \the\year\ \thanklessauthor

\par\smallcaps{Published by \thanklesspublisher}

\par{github.com/tevaughan/coordinate-systems}

\par Permission is granted to copy, distribute and/or modify this document
   under the terms of the {\it GNU Free Documentation License,} Version 1.3 or
   any later version published by the Free Software Foundation; with no
   Invariant Sections, no Front-Cover Texts, and no Back-Cover Texts.  A copy
   of the license is included in the Chapter, ``GNU Free Documentation
   License.''

%\par\textit{First printing, \monthyear}
\end{fullwidth}

% r.5 contents
\tableofcontents

%\listoffigures

%\listoftables

% r.7 dedication
\cleardoublepage
~\vfill
\begin{doublespace}
\noindent\fontsize{18}{22}\selectfont\itshape
\nohyphenation
Dedicated to \mbox{Neil Huffaker}, whose physical approach to vector
   mathematics is fascinating, and to \mbox{Jack Cohn}, whose zeal for the
   finer points of mathematical physics inspired me to learn some of them.
\end{doublespace}
\vfill
\vfill


% r.9 introduction
\cleardoublepage
\chapter*{Prolegomenon}

In the present book I work through the basics of differential geometry.  Also,
for coordinate systems and component transformations in Euclidean space, I
develop an approach that explicitly treats a certain kind of vector---for
example, one that might be chosen as a basis vector---as independent of any
coordinate system.  This is inspired by ideas that Neil Huffaker introduced in
his graduate lectures on electrodynamics at the University of Oklahoma in the
early 1990s.  The material should be accessible to anyone who has studied
calculus-based physics.

The form of the book follows the style of Edward R.~Tufte and Richard Feynman
\citep{pkg-tufte}.

The title is ``Coordinate Systems in Physical Space,'' and a brief reflection
on the choice of title provides an introduction to the material.

\section{Coordinate Systems}

First of all, the book is about coordinate systems.  But what does
``coordinate'' mean?

\subsection{Coordinate}

In mathematics, a coordinate is typically a real number that constrains the
location of a point in a \emph{manifold}.  A manifold, in turn, is an
$n$-dimensional space that resembles Euclidean space at least in a small region
near every point in the space.\footnote{%
   We shall typically consider cases in which $n \in \{1, 2, 3, 4\}$.%
}
From now on, we shall not use the word ``manifold'' but instead just use the
word ``space'' to mean the same thing: a \emph{locally Euclidean, mathematical
space}.  (We shall, however, distinguish mathematical space from physical space
below.)

\subsection{Coordinate Tuple and Coordinate System}

A point in an $n$-dimensional space is uniquely identified by an $n$-tuple, an
ordered set of $n$ coordinates.  The \emph{coordinate system} is a continuous,
one-to-one function that maps from the set of coordinate $n$-tuples onto the
points in the space.\footnote{%
   For the function to be continuous, an infinitesimal difference between
   values of any coordinate must never produce a finite distance between the
   corresponding points.  The one-to-one nature of the mapping means that for
   every point in the space, there is exactly one tuple corresponding to it.%
}
Suppose that the first coordinate in the tuple is $x_1$, and the last is $x_n$.
Then we can specify any coordinate as $x_i$, where $i \in \{1, 2, 3, \ldots,
n\}$.

\subsection{Coordinate Line}

Allowing the $i$th coordinate to vary, while holding every other coordinate at
fixed value, produces tuples for points along a \emph{coordinate line} for the
$i$th coordinate.

For example, the surface of a sphere is a two-dimensional space.  Each of the
latitude and the longitude is a coordinate.  These coordinates form a
two-tuple, or ordered pair, $(\alpha,\beta)$, where $\alpha$ is the latitude
and $\beta$ the longitude.  By holding the latitude fixed, one finds that the
set of longitudes maps to a parallel, a coordinate line for the longitude, on
the surface of the sphere.  See Figure~\ref{fig:sphere}.  Every path of
constant latitude on the sphere is called ``a parallel'' because no parallel
intersects any other parallel.  Among the parallels, only the equator is a
great circle; that is, the parallel at zero latitude is the only parallel
circle centered on the center of the sphere.  Every path of constant
longitude---that is, every coordinate line for the latitude---is called ``a
meridian'' and intersects every other meridian, both at the north pole and at
the south pole.  Every meridian is a great circle.

\begin{marginfigure}
   \begin{center}
   \begin{asy}
      defaultpen(fontsize(8pt));
      settings.prc = false;
      size(4cm,0);
      import three;
      draw(unitsphere, white);
      path3 equator = circle(c=O, r=1.001, normal=Z);
      path3 par     = circle(c=O+1/sqrt(2)*Z, r=1/sqrt(2)+0.001, normal=Z);
      path3 prime   = circle(c=O, r=1.001, normal=X);
      path3 mer     = circle(c=O, r=1.001, normal=cos(30)*X+sin(30)*Y);
      draw(equator, red);
      draw(par, red);
      draw(prime, yellow);
      draw(mer, yellow);
   \end{asy}
   \end{center}
   \caption{%
      Coordinate lines on the surface of a sphere. The equator and a northern
      parallel appear in red. Two meridians appear in yellow.%
   }
   \label{fig:sphere}
\end{marginfigure}

\subsection{Value of Coordinate}

In my treatment, a coordinate is not necessarily a real number.  It might
instead be an irreducibly physical quantity, like a length.  A physical
quantity is not a number, but it can be expressed as the product of a number
and a standard unit, such as an inch.  What is generally required is that a
coordinate be an element of a simply ordered set.\footnote{%
   A set $S$ is simply ordered if, for every $a, b, c \in S$,
   \begin{enumerate}
      \item $a \leq b$ and $b \leq a$ imply that $a = b$;
      \item $a \leq b$ and $b \leq c$ imply that $a \leq c$; and
      \item either $a \leq b$ or $b \leq a$.
   \end{enumerate}%
}
The real numbers are simply ordered, and so is the set of all lengths of the
form, $x = r \; \text{in}$, where $r$ is a real number, and ``$\text{in}$''
represents an inch.

\subsection{Arbitrary Choice of Coordinate System}

No particular coordinate system need be chosen for a given space, and so the
choice of coordinate system is arbitrary.  However, one coordinate system might
fit a space more naturally than another would fit it.  For example, the most
natural coordinate systems on the surface of an oblate
spheroid\footnote[][-0.9in]{%
   There is no preferred orientation for a grid of latitude and longitude on a
   perfect sphere.  An oblate spheroid is obtained by a scaling transformation
   that reduces the size of the sphere along a single direction.  The shape of
   the Earth is approximated by an oblate spheroid better than by a perfect
   sphere.%
}
would have the equator on the longest circumference of the spheroid.

\subsection{Cartesian Coordinate System}

There is a special kind of coordinate system, a \emph{Cartesian} coordinate
system.  In a Cartesian coordinate system, the shortest path between every pair
of points on every coordinate line lies on that coordinate line.  Not every
space admits a Cartesian coordinate system.  For example, the surface of a
sphere does not admit a Cartesian coordinate system.\footnote{%
   The shortest path between two distinct points along a parallel does not lie
   along the parallel.  The shortest path always lies along a great circle, and
   a parallel is usually not a great circle.%
}
In fact, only a \emph{Euclidean} space admits a Cartesian coordinate system.
We shall see that what makes a Euclidean space special is that it has zero
\emph{curvature}.

\subsection{Transformation of Coordinate System}

We shall spend some time discussing the difference between a quantity that
depends on the choice of coordinate system and a quantity that does not depend
on the choice.  For a quantity that does depend on the coordinate system, we
shall explore, particularly for Cartesian coordinates, the different ways in
which a quantity can transform as the coordinate system changes.

\section{Physical Space}

My treatment will involve two key principles.  First, a distance is not reduced
to a number; rather, distance remains an intuitive, undefined, physical,
non-numeric quantity, expressed in terms of a fundamental unit, such as the
inch or the meter.  Nevertheless, a distance can be multiplied by a number.
Second, neither a point in space nor a displacement vector between two points
is reduced to a tuple of quantities.  Rather, each of the point and the vector
is regarded as both simple\footnote{%
   What is simple is not composed of parts; that is, not complex.  A tuple of
   coordinates is complex in the sense that it has parts.  Further, a
   coordinate system may map this tuple to a point in a space, but the point
   itself has no parts.  The point is simply a location.%
}
and logically prior to any coordinate system.

As a global physical hypothesis, Euclidean geometry has been ruled out by
experiment.\footnote{%
   Experiments in lunar ranging from the surface of the Earth and data from the
   spacecraft, Gravity Probe B, have directly and repeatably measured the
   curvature of space near the Earth. The results are inconsistent with
   Euclidean physical space.  \citep{gpb-2011, miller-geodetic}
}
However, even in the non-Euclidean geometries of modern physics, the Euclidean
ideal is a better and better approximation as the size of the region of space
under consideration becomes smaller and smaller.  While Euclidean geometry as a
physical hypothesis applied globally has been ruled out by experiment,
Euclidean geometry as a physical hypothesis applied separately to every
sufficiently small region of physical space has not been ruled out.  In fact,
the space of general relativity is called ``a locally Euclidean space.''

Anyway, most of the narrative of the book is set in the practically Euclidean,
three-dimensional space of ordinary experience, but the non-Euclidean,
two-dimensional space of the surface of the sphere is also explored.



%%
% Start the main matter (normal chapters)
\mainmatter

\chapter{Space}
\label{ch:space}

\newthought{A point}, as introduced in high-school geometry, is a particular,
pure location, without any extent: no length, no width, no height.

\newthought{Every other geometric thing}, such as the line, the plane, and the
space, is introduced as merely a collection of points.

\bigskip
Yet there is a problem that comes from imagining, say, a line segment as merely
a collection of points.  A line segment has length, but a point has no length.
No point can touch another distinct point, and so points cannot simply lie in
unbroken contact, such that their lengths add up to that of the segment.  In
high-school physics, the student learns to do mathematical calculation with
physical units.  One cannot add two quantities together if one have a unit of a
kind different from that of the other quantity.\sidenote{%
   For example, one cannot add a two kilograms and three meters.}
A point has no extent in meters, and so one cannot construct a line segment
three meters long merely by adding points together.

The solution to this problem is found in the calculus.  The length along a
segment of a curve is the limit of an ever-increasing number of ever-smaller
straight lengths along the curve.  (Such an approach can be used to find the
length of a straight-line segment, too.)  The segment is divided into an
infinite number of things, but each of these things is a length, not a point.
Even in the limit, each piece still has some length, though infinitesimally
small, and so the pieces can be added together to produce a finite length.

Similarly, a surface must be composed of bits with area, and a space must be
composed of bits with volume.

\section{Distance and Path}

\section{Geodesic}

\appendix

%%
% The back matter contains appendices, bibliographies, indices, glossaries,
% etc.
\backmatter

\input{fdl-1.3}

%\bibliography{sample-handout}
%\bibliographystyle{plainnat}

\printindex

\end{document}

