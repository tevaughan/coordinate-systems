
% vim: set filetype=tex:

% Prints an epigraph and speaker in sans serif, all-caps type.
\newcommand{\openepigraph}[2]{%
  {
  \sffamily\large
  \noindent\textit{#1} % epigraph
  \ \\%
  \ \\%
  \noindent{#2} % author
  }
}

% Generates the index
\usepackage{makeidx}
\makeindex

\usepackage{amsmath} % Allow use of $\text{}$.
\usepackage{amsthm} % theorem
\usepackage{amsfonts}
\usepackage[greek,english]{babel} % language selection
\usepackage{booktabs} % \toprule, \midrule, \bottomrule
\usepackage[margin=10pt,font={sl},labelfont=bf]{caption} % figure captions
\usepackage{fancyhdr}
\usepackage{framed} % environment for framed box
\usepackage{graphicx}
\usepackage{natbib}
\usepackage[rgb]{xcolor} % might need loading before asymptote
\usepackage{asymptote} % figures
\usepackage{times}
\usepackage{vmargin}

% must be last
\usepackage[colorlinks=true,citecolor=blue,hyperfootnotes=false]{hyperref}

% uncomment if you prefer colored hyperlinks (e.g., for onscreen viewing)
\hypersetup{colorlinks}

% Babel setup. I want the polutoniko version of Greek for the rich set of
% accents needed properly to quote from John's gospel.
\languageattribute{greek}{polutoniko}

% Amsthm setup.
% Empty argument to newtheoremstyle leaves default.
\newtheoremstyle{mytheorem}%                   name
{\topsep}%                                     space above
{\topsep}%                                     space below
{\slshape}%                                    body font
{0pt}%                                         indent
{\bfseries}%                                   head font
{\ }%                                          head punctuation
{5pt plus 1pt minus 1pt}%                      head space
{\thmname{#1}\thmnumber{\ #2}:\thmnote{\ #3}}% head spec
\theoremstyle{mytheorem}
\newtheorem{exercise}{Exercise}[section]
\newtheorem{definition}{Definition}[chapter]
\newtheorem{postulate}{Postulate}[chapter]
\newtheorem{lemma}{Lemma}[chapter]
\newtheorem{theorem}{Theorem}[chapter]

\selectlanguage{english} % Configure babel.

% vmargin setup
\setpapersize{USletter}
\setmarginsrb%
{0.375in}%           left
{0.375in}%           top
{0.375in}%           right
{0.5in}%             bottom
{3\baselineskip}%    headheight
{2\baselineskip}%    headsep
{3\baselineskip}%    footheight
{4\baselineskip}%    footskip

% mydate macro
\newcommand{\mydate}{%
   \number\year\space%
   \ifcase\month\or%
      Jan\or\ Feb\or\ Mar\or\ Apr\or\ May\or\ Jun\or%
      Jul\or\ Aug\or\ Sep\or\ Oct\or\ Nov\or\ Dec
   \fi\space%
   \number\day%
}

\newcommand{\doctitle}{Coordinate Systems in Physical Space}
\newcommand{\docsubtitle}{}
\newcommand{\docauthor}{Thomas E.~Vaughan}
\newcommand{\docsite}{\url{http://github.com/tevaughan/coordinate-systems}}

