
% vim: set filetype=tex expandtab shiftwidth=3:

\frontmatter

% r.1 blank page
\blankpage

% v.2 epigraphs
\newpage\thispagestyle{empty}
\openepigraph{%
Wheeeeeeeeeeeeeeeeeeee.
}{%
Neil Huffaker, when writing an interesting mathematical result onto the board
during a lecture
}
\vfill
\openepigraph{%
Well, that's unfortunate.
}{%
Jack Cohn, when---after a long, difficult, and messy attempt at a derivation on
the board---arriving at the wrong result during a lecture
}
\vfill

% r.3 full title page
\maketitle

% v.4 copyright page
\newpage
\begin{fullwidth}
~\vfill
\thispagestyle{empty}
\setlength{\parindent}{0pt}
\setlength{\parskip}{\baselineskip}
Copyright \copyright\ \the\year\ \thanklessauthor

\par\smallcaps{Published by \thanklesspublisher}

\par{github.com/tevaughan/coordinate-systems}

\par Permission is granted to copy, distribute and/or modify this document
   under the terms of the {\it GNU Free Documentation License,} Version 1.3 or
   any later version published by the Free Software Foundation; with no
   Invariant Sections, no Front-Cover Texts, and no Back-Cover Texts.  A copy
   of the license is included in the Chapter, ``GNU Free Documentation
   License.''

%\par\textit{First printing, \monthyear}
\end{fullwidth}

% r.5 contents
\tableofcontents

%\listoffigures

%\listoftables

% r.7 dedication
\cleardoublepage
~\vfill
\begin{doublespace}
\noindent\fontsize{18}{22}\selectfont\itshape
\nohyphenation
Dedicated to \mbox{Neil Huffaker}, whose physical approach to vector
   mathematics is fascinating, and to \mbox{Jack Cohn}, whose zeal for the
   finer points of mathematical physics inspired me to learn some of them.
\end{doublespace}
\vfill
\vfill


% r.9 introduction
\cleardoublepage
\chapter*{Prolegomenon}

In the present book, I introduce the idea of a coordinate system and discuss
two different kinds of quantity:
\begin{enumerate}
   \item that which is independent of the choice of coordinate system and
   \item that which depends on the choice of coordinate system.
\end{enumerate}
I focus mostly on the three-dimensional, physical space of ordinary experience.

The subject matter touches on concepts in differential geometry and is aimed at
the development of a consistent notation.  I define what a coordinate system is
and, for a change of coordinate system, explore the transformation of a
quantity that depends on the choice of coordinates.  I have been dissatisfied
with the traditional approach and should like to expose ideas that Neil
Huffaker introduced in his graduate lectures on electrodynamics at the
University of Oklahoma in the early 1990s.  The material should be accessible
to anyone who has studied calculus-based physics.

The format of the book follows the style of Edward R.~Tufte and Richard
Feynman \citep{pkg-tufte}.

The treatment below involves two key principles.  First, a distance is not
reduced to a number; rather, distance remains an intuitive, undefined,
physical, non-numeric quantity, expressed in terms of a fundamental unit, such
as the inch or the meter.  Nevertheless, a distance can be multiplied by a
number.  Second, neither a point in space nor a displacement vector between two
points is reduced to an ordered set of quantities.  Rather, each of the point
and the vector is regarded as both simple and logically prior to any coordinate
system.

As a physical hypothesis, Euclidean geometry has been ruled out by
experiment.\sidenote{%
   At least a couple of experiments, including lunar ranging and Gravity Probe
   B, have directly and repeatably measured the curvature of space near the
   Earth. The results are inconsistent with Euclidean physical space.
   \citep{gpb-2011, miller-geodetic}
}
Whereas a purely mathematical idea cannot be ruled out by experiment, a
physical hypothesis can be ruled out.  What keeps an hypothesis from being
purely mathematical is its reference to concrete physical quantities, such as
distance, rather than only to abstractions, such as number.\sidenote{%
   In an explicitly physical treatment, a number multiplies the physical
   quantity in order to scale it.}

Euclidean geometry can, in a sense, be regarded as purely mathematical.  It
remains valid as an artifact of mathematical reasoning applied to a small set
of axioms, and it can be disconnected from physical space.  For example,
distance can be made abstract, as by reducing it to the idea of number.

What is not purely mathematical is the hypothesis according to which Euclid
perfectly describes the geometry of physical space.  \emph{That} hypothesis has
been tested and found wanting.  To apply Euclidean geometry to physical space
is to contradict the results of experiment.

However, even in the non-Euclidean geometries of modern physics, the Euclidean
ideal is a better and better approximation as the size of the region of space
under consideration becomes smaller and smaller.  While Euclidean geometry as a
physical hypothesis applied globally has been ruled out by experiment,
Euclidean geometry as a physical hypothesis applied separately to every
sufficiently small region of physical space has not been ruled out.  In fact,
the space of general relativity is called ``a locally Euclidean space.''

Anyway, most of the narrative of the book is set in the practically Euclidean,
three-dimensional space of ordinary experience, but the non-Euclidean,
two-dimensional space of the surface of the sphere is also explored.

