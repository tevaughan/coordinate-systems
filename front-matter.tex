
% vim: set filetype=tex:

\frontmatter

% r.1 blank page
\blankpage

% v.2 epigraphs
\newpage\thispagestyle{empty}
\openepigraph{%
Wheeeeeeeeeeeeeeeeeeee.
}{%
Neil Huffaker, on arriving at a conclusion at the board during a lecture
}
\vfill
\openepigraph{%
Well, that's unfortunate.
}{%
Jack Cohn, after deriving the wrong result at the board during a lecture
}

% r.3 full title page
\maketitle

% v.4 copyright page
\newpage
\begin{fullwidth}
~\vfill
\thispagestyle{empty}
\setlength{\parindent}{0pt}
\setlength{\parskip}{\baselineskip}
Copyright \copyright\ \the\year\ \thanklessauthor

\par\smallcaps{Published by \thanklesspublisher}

\par{github.com/tevaughan/coordinate-systems}

\par Permission is granted to copy, distribute and/or modify this document
   under the terms of the {\it GNU Free Documentation License,} Version 1.3 or
   any later version published by the Free Software Foundation; with no
   Invariant Sections, no Front-Cover Texts, and no Back-Cover Texts.  A copy
   of the license is included in the Chapter, ``GNU Free Documentation
   License.''

%\par\textit{First printing, \monthyear}
\end{fullwidth}

% r.5 contents
\tableofcontents

%\listoffigures

%\listoftables

% r.7 dedication
\cleardoublepage
~\vfill
\begin{doublespace}
\noindent\fontsize{18}{22}\selectfont\itshape
\nohyphenation
Dedicated to those who appreciate mathematical physics, to \mbox{Neil
   Huffaker}, whose approach to vector mathematics was fascinating, and to
   \mbox{Jack Cohn}, whose zeal for teaching mathematical methods in physics
   inspired my desire to learn.
\end{doublespace}
\vfill
\vfill


% r.9 introduction
\cleardoublepage
\chapter*{Prolegomenon}

Herebelow is an introduction to the idea of a coordinate system.  Not purely
mathematical, this treatment involves at least two key elements of physical
theory.  In the first place, the idea of distance is not reduced to the idea of
number.  Rather, distance remains an intuitive, undefined, physical,
non-numeric quantity.  Nevertheless, this quantity can be multiplied by a
number.  In the second place, neither a location in space nor a displacement
vector between two points in space is reduced to an ordered set of quantities.
Rather, each of the location and the vector is regarded as logically prior to
and independent of any coordinate system.

