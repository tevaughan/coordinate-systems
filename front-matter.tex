
% vim: set filetype=tex:

\frontmatter

% r.1 blank page
\blankpage

% v.2 epigraphs
\newpage\thispagestyle{empty}
\openepigraph{%
Wheeeeeeeeeeeeeeeeeeee.
}{%
Neil Huffaker, on arriving at a conclusion at the board during a lecture
}
\vfill
\openepigraph{%
Well, that's unfortunate.
}{%
Jack Cohn, after deriving the wrong result at the board during a lecture
}

% r.3 full title page
\maketitle

% v.4 copyright page
\newpage
\begin{fullwidth}
~\vfill
\thispagestyle{empty}
\setlength{\parindent}{0pt}
\setlength{\parskip}{\baselineskip}
Copyright \copyright\ \the\year\ \thanklessauthor

\par\smallcaps{Published by \thanklesspublisher}

\par{github.com/tevaughan/coordinate-systems}

\par Permission is granted to copy, distribute and/or modify this document
   under the terms of the {\it GNU Free Documentation License,} Version 1.3 or
   any later version published by the Free Software Foundation; with no
   Invariant Sections, no Front-Cover Texts, and no Back-Cover Texts.  A copy
   of the license is included in the section entitled ``GNU Free Documentation
   License.''

%\par\textit{First printing, \monthyear}
\end{fullwidth}

% r.5 contents
\tableofcontents

\listoffigures

\listoftables

% r.7 dedication
\cleardoublepage
~\vfill
\begin{doublespace}
\noindent\fontsize{18}{22}\selectfont\itshape
\nohyphenation
Dedicated to those who appreciate \LaTeX{} 
and the work of \mbox{Edward R.~Tufte} 
and \mbox{Donald E.~Knuth}.
\end{doublespace}
\vfill
\vfill


% r.9 introduction
\cleardoublepage
\chapter*{Prolegomenon}

Herebelow is an introduction to the idea of a coordinate system.  Not purely
mathematical, this treatment involves elements of physical theory.  In
particular, the idea of distance is not reduced to the idea of number.  Rather,
distance remains an undefined, physical, non-numeric quantity.  Nevertheless,
this quantity can be multiplied by a number.

