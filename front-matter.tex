
% vim: set filetype=tex expandtab shiftwidth=3:

\frontmatter

% r.1 blank page
\blankpage

% v.2 epigraphs
\newpage\thispagestyle{empty}
\openepigraph{%
Wheeeeeeeeeeeeeeeeeeee.
}{%
Neil Huffaker, when writing an interesting mathematical result onto the board
during a lecture
}
\vfill
\openepigraph{%
Well, that's unfortunate.
}{%
Jack Cohn, when---after a long, difficult, and messy attempt at a derivation on
the board---arriving at the wrong result during a lecture
}
\vfill

% r.3 full title page
\maketitle

% v.4 copyright page
\newpage
\begin{fullwidth}
~\vfill
\thispagestyle{empty}
\setlength{\parindent}{0pt}
\setlength{\parskip}{\baselineskip}
Copyright \copyright\ \the\year\ \thanklessauthor

\par\smallcaps{Published by \thanklesspublisher}

\par{github.com/tevaughan/coordinate-systems}

\par Permission is granted to copy, distribute and/or modify this document
   under the terms of the {\it GNU Free Documentation License,} Version 1.3 or
   any later version published by the Free Software Foundation; with no
   Invariant Sections, no Front-Cover Texts, and no Back-Cover Texts.  A copy
   of the license is included in the Chapter, ``GNU Free Documentation
   License.''

%\par\textit{First printing, \monthyear}
\end{fullwidth}

% r.5 contents
\tableofcontents

%\listoffigures

%\listoftables

% r.7 dedication
\cleardoublepage
~\vfill
\begin{doublespace}
\noindent\fontsize{18}{22}\selectfont\itshape
\nohyphenation
Dedicated to \mbox{Neil Huffaker}, whose physical approach to vector
   mathematics is fascinating, and to \mbox{Jack Cohn}, whose zeal for the
   finer points of mathematical physics inspired me to learn some of them.
\end{doublespace}
\vfill
\vfill


% r.9 introduction
\cleardoublepage
\chapter*{Prolegomenon}

\begin{marginfigure}
   \begin{center}
   \begin{asy}
      defaultpen(fontsize(8pt));
      settings.prc = false;
      size(2cm,0);
      import three;
      draw(unitsphere, white);
      path3 equator = circle(c=O, r=1.001, normal=Z);
      path3 par     = circle(c=O+1/sqrt(2)*Z, r=1/sqrt(2)+0.001, normal=Z);
      path3 prime   = circle(c=O, r=1.001, normal=X);
      path3 mer     = circle(c=O, r=1.001, normal=cos(30)*X+sin(30)*Y);
      draw(equator, red);
      draw(par, red);
      draw(prime, yellow);
      draw(mer, yellow);
   \end{asy}
   \end{center}
   \caption{%
      Every path of constant latitude on the sphere is called ``a parallel''
      because no parallel intersects any other parallel.  Among the parallels,
      only the equator is a great circle; that is, the parallel at zero
      latitude is the only parallel circle centered on the center of the
      sphere.  The equator and a northern parallel are shown.  In contrast,
      every path of constant longitude is called ``a meridian'' and intersects
      every other meridian, both at the north pole and at the south pole.
      Also, every meridian is a great circle.  Two meridians are shown.%
   }
   \label{fig:sphere}
\end{marginfigure}

The present work is motivated by two principal desires of mine: first, to work
through the basics of differential geometry, at least in a few simple cases,
and, second, to expose ideas that Neil Huffaker introduced in his graduate
lectures on electrodynamics at the University of Oklahoma in the early 1990s.
Huffaker's approach to non-orthonormal coordinate systems in Euclidean space
has long struck me as interesting enough to write up.  The material should be
accessible to anyone who has studied calculus-based physics.

The form of the book follows the style of Edward R.~Tufte and Richard Feynman
\citep{pkg-tufte}.

The title of the book is ``Coordinate Systems in Physical Space,'' and a brief
reflection on the choice of words provides an introduction to the material.

\section{Coordinate}

A \emph{coordinate} is an element of a simply ordered set,\sidenote[][0.1in]{%
   A set $S$ is simply ordered if, for every $a, b, c \in S$,
   \begin{enumerate}
      \item $a \leq b$ and $b \leq a$ imply that $a = b$;
      \item $a \leq b$ and $b \leq c$ imply that $a \leq c$; and
      \item either $a \leq b$ or $b \leq a$.
   \end{enumerate}%
}
which is mapped continuously and in a one-to-one way onto a path in a
space.\sidenote{%
   A mapping implies the existence of a function that takes every coordinate to
   a corresponding point on the path.  For the function to be continuous, an
   infinitesimal change in coordinate must never produce a finite change in the
   distance along the path between mapped points.  The one-to-one nature of the
   mapping means that for every point along the path, there is exactly one
   coordinate corresponding to it.%
}
In mathematics, a coordinate is typically a real number.  For example, on the
surface of a sphere, each of the latitude and the longitude is a coordinate.
By holding the latitude fixed, one finds that the set of longitudes maps to a
parallel, which is a path on the surface of the sphere.  See
Figure~\ref{fig:sphere}.  In my treatment, which is physical and not purely
mathematical, a coordinate is often not a number but a \emph{length}, an
irreducibly physical quantity, which can be expressed as the product of a
number and a standard unit (such as an inch).

\section{Coordinate System}

A \emph{coordinate tuple} is an ordered set of coordinates.  For example,
$(\alpha,\beta)$, where $\alpha$ is the latitude and $\beta$ the longitude, is
a coordinate tuple identifying a point on the surface of a sphere.  A
\emph{coordinate system} is a continuous, one-to-one, and onto mapping from a
set of coordinate tuples to the points in a space.

No particular coordinate system need be chosen for a given space, and so the
choice of coordinate system is arbitrary.  However, one coordinate system might
fit a space more naturally than another would fit it.  Also, there is a special
kind of coordinate system, a \emph{Cartesian} coordinate system; not every
space admits a Cartesian coordinate system.  In fact, only a \emph{Euclidean}
space admits a Cartesian coordinate system.

By defining a \emph{geodesic} in a space as the shortest path between two
points, we shall show that changing any single Cartesian coordinate while
holding the others fixed will produce a mapping onto a geodesic in the space.
We shall also explore the idea of \emph{curvature} and show that a Euclidean
space has zero curvature.

We shall spend some time discussing the difference between a quantity that
depends on the choice of coordinate system and a quantity that does not depend
on the choice.  For a quantity that does depend on the coordinate system, we
shall explore, particularly for Cartesian coordinates, the different ways in
which a quantity can transform as the coordinate system changes.

\section{Physical Space}

My treatment will involve two key principles.  First, a distance is not reduced
to a number; rather, distance remains an intuitive, undefined, physical,
non-numeric quantity, expressed in terms of a fundamental unit, such as the
inch or the meter.  Nevertheless, a distance can be multiplied by a number.
Second, neither a point in space nor a displacement vector between two points
is reduced to a tuple of quantities.  Rather, each of the point and the vector
is regarded as both simple\footnote{%
   What is simple is not composed of parts; that is, not complex.  A tuple of
   coordinates is complex in the sense that it has parts.  Further, a
   coordinate system may map this tuple to a point in a space, but the point
   itself has no parts.  The point is simply a location.%
}
and logically prior to any coordinate system.

As a global physical hypothesis, Euclidean geometry has been ruled out by
experiment.\sidenote{%
   Experiments in lunar ranging from the surface of the Earth and data from the
   spacecraft, Gravity Probe B, have directly and repeatably measured the
   curvature of space near the Earth. The results are inconsistent with
   Euclidean physical space.  \citep{gpb-2011, miller-geodetic}
}
However, even in the non-Euclidean geometries of modern physics, the Euclidean
ideal is a better and better approximation as the size of the region of space
under consideration becomes smaller and smaller.  While Euclidean geometry as a
physical hypothesis applied globally has been ruled out by experiment,
Euclidean geometry as a physical hypothesis applied separately to every
sufficiently small region of physical space has not been ruled out.  In fact,
the space of general relativity is called ``a locally Euclidean space.''

Anyway, most of the narrative of the book is set in the practically Euclidean,
three-dimensional space of ordinary experience, but the non-Euclidean,
two-dimensional space of the surface of the sphere is also explored.

