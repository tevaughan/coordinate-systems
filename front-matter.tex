
% vim: set filetype=tex:

\frontmatter

% r.1 blank page
\blankpage

% v.2 epigraphs
\newpage\thispagestyle{empty}
\openepigraph{%
Wheeeeeeeeeeeeeeeeeeee.
}{%
Neil Huffaker, when writing an interesting mathematical result onto the board
during a lecture
}
\vfill
\openepigraph{%
Well, that's unfortunate.
}{%
Jack Cohn, when---after a long, difficult, and messy attempt at a derivation on
the board---arriving at the wrong result during a lecture
}
\vfill

% r.3 full title page
\maketitle

% v.4 copyright page
\newpage
\begin{fullwidth}
~\vfill
\thispagestyle{empty}
\setlength{\parindent}{0pt}
\setlength{\parskip}{\baselineskip}
Copyright \copyright\ \the\year\ \thanklessauthor

\par\smallcaps{Published by \thanklesspublisher}

\par{github.com/tevaughan/coordinate-systems}

\par Permission is granted to copy, distribute and/or modify this document
   under the terms of the {\it GNU Free Documentation License,} Version 1.3 or
   any later version published by the Free Software Foundation; with no
   Invariant Sections, no Front-Cover Texts, and no Back-Cover Texts.  A copy
   of the license is included in the Chapter, ``GNU Free Documentation
   License.''

%\par\textit{First printing, \monthyear}
\end{fullwidth}

% r.5 contents
\tableofcontents

%\listoffigures

%\listoftables

% r.7 dedication
\cleardoublepage
~\vfill
\begin{doublespace}
\noindent\fontsize{18}{22}\selectfont\itshape
\nohyphenation
Dedicated to \mbox{Neil Huffaker}, whose physical approach to vector
   mathematics is fascinating, and to \mbox{Jack Cohn}, whose zeal for the
   finer points of mathematical physics inspired me to learn some of them.
\end{doublespace}
\vfill
\vfill


% r.9 introduction
\cleardoublepage
\chapter*{Prolegomenon}

The present book is an introduction to the idea of a coordinate system in the
three-dimensional, physical space of ordinary experience.

The subject matter touches on concepts in differential geometry but is aimed at
the development of a consistent notation, both for the definition of a
coordinate system and for the transformation of the coordinate-dependent
components of a coordinate-free quantity.  I have been dissatisfied with the
traditional approach and should like to expose ideas that Neil Huffaker
introduced in his graduate lectures on electrodynamics at the University of
Oklahoma in the early 1990s.  The material should be accessible to anyone who
has studied calculus-based physics.

The format of the book follows the style of Edward R.~Tufte and Richard
Feynman\cite[-0.5in]{pkg-tufte}.

Not purely mathematical, the treatment below involves at least two key elements
of physical theory.  In the first place, the idea of distance is not reduced to
the idea of number.  Rather, distance remains an intuitive, undefined,
physical, non-numeric quantity, expressed in terms of a fundamental unit, such
as the inch or the meter.  Nevertheless, this quantity can be multiplied by a
number.  In the second place, neither a location in space nor a displacement
vector between two points in space is reduced to an ordered set of quantities.
Rather, each of the location and the vector is regarded as simple, logically
prior to any coordinate system, and independent of it.

Like physics in general, the treatment is mathematical but not entirely of
mathematics: Location, distance, and direction are physical realities amenable
to mathematical description. Such a description can be tested by experiment,
and some of the descriptions presented below have been ruled out by
experiment.\sidenote[][-2.5in]{%
   The Newtonian hypothesis, according to which the physical space-time of
   ordinary experience is Euclidean, is ruled out by astronomical observation.
   The occasional Newtonian prediction, of the observed result of an experiment
   testing the effect of gravity, fails repeatably.  The measurement of the
   deflection of the apparent direction toward a star near the Sun during an
   eclipse in 1919 provides the initial example.  To the contrary, the
   Einsteinian hypothesis, according to which space-time is non-Euclidean in a
   certain way, has (so far) survived every experimental challenge.  Because it
   is a scientific theory referring to things (like space-time curvature) that
   can never be directly perceived by the senses, the Einsteinian hypothesis
   can never be proved true, but the Newtonian hypothesis---and arguably every
   hypothesis asserting the Euclidean nature of physical space-time---has been
   ruled out.
}
Whereas a purely mathematical development cannot be ruled out by experiment, a
mathematical hypothesis about physical reality can be ruled out.  What keeps
the hypothesis from being purely mathematical is its reference to concrete
physical quantities, such as distance, rather than only to abstractions, such
as number, which in an explicitly physical treatment multiplies the physical
quantity in order to scale it.

David E.~Rowe\cite{Rowe2006} argues that one of the reasons why non-Euclidean
geometry was difficult to accept at the beginning of the 20th Century is that
geometry---Euclidean geometry---had long been commonly regarded as purely
mathematical.  So no experiment could ever contradict it.

While Euclidean geometry can in some sense be regarded as purely mathematical,
what is not purely mathematical is the hypothesis according to which Euclid
perfectly describes the geometry of physical space.  \emph{That} hypothesis has
been tested and found wanting.  Euclidean geometry remains valid as an artifact
of pure, mathematical reasoning from a small set of axioms.  However, when a
distance in Euclidean geometry is interpreted as a physical distance, at least
one of the axioms of Euclidean geometry is now known to be false.  Even in the
non-Euclidean geometries of modern physics, the Euclidean ideal is a better and
better approximation as the size of the region of space under consideration
becomes smaller and smaller.  While Euclidean geometry as a physical hypothesis
applied globally has been ruled out by experiment, Euclidean geometry as a
physical hypothesis applied separately to every local region has not been ruled
out.  That is why the space of general relativity is called ``a locally
Euclidean space.''

Most of the narrative of the book is set in the three-dimensional space of
ordinary experience, but the non-Euclidean, two-dimensional space of the
surface of the sphere is also explored.

